\documentclass[a4paper,12pt]{article}

\usepackage[pdftex]{graphicx}
\usepackage{verbatim}
\usepackage{url}

\usepackage[spanish]{babel}
\usepackage{makeidx}
\usepackage{fancybox}
\usepackage{fancyhdr}
\usepackage[svgnames]{xcolor}
\usepackage{pgf}
\usepackage{listings}

\begin{document}
\lstset{basicstyle=\small\ttfamily}
\lstset{keywordstyle=\color{Navy}}
\lstset{identifierstyle=\color{Brown}}
\lstset{commentstyle=\color{Green}}
\lstset{showstringspaces=false}
\lstset{captionpos=b}

\title{PlugCode 1.0alpha Reference Guide}
\author{Sergio Davis}
\date{\today}

\maketitle

\section{What is PlugCode?}

PlugCode it is a \textit{dialect} of C programing language, developed to write
\textit{plug-ins} of \textbf{lpmd} software package. PlugCode it is a
\emph{superset}  of C, this mean that any C routine it is allowed
automatically as a valid PlugCode subroutine.

PlugCode add C facilities as macros, clousures (in a limited way), and allow a
very clear aand optimized syntaxis over vector operations, without the poor
performance usually related to high level languages like C++. These
characteristics are possible because PlugCode is internally tranformed to
traditional C code\footnotemark ~ during compiling process.

\footnotetext{To be accurate, traditional C means the standard C99, that
incorporate comments with \texttt{//} and declare allowed inside \texttt{for}
and \texttt{while} cicles.}

\begin{figure}[ht]\centering\shadowbox{
\begin{minipage}{12cm}\scriptsize
\lstset{language=C}
\begin{lstlisting}
@slot Evaluate (reader)
{
 Vector * vel = GetArray("vel");
 double ti = 0.0e0;
 for (long i=0;i<size;i++)
 {
  ti += 0.5*M*SquareModule(vel[i]);
 }
 ti *= KIN2EV;
 ti /= ((3.0/2.0)*KBOLTZMANN*totalsize);
 return ti;
}
\end{lstlisting}
\end{minipage}}
\caption{``slot'' example wrote at PlugCode.}
\label{fig_code_slot}
\end{figure}

\begin{figure}[ht]\centering\shadowbox{
\begin{minipage}{12cm}\scriptsize
\lstset{language=C}
\lstset{basicstyle=\tiny\ttfamily}
\begin{lstlisting}
void Evaluate(const RawAtomSet * aset, const RawCell * cell)
{
 double * vel = NULL;
 ASet_GetArrays(aset, NULL, &vel, NULL, NULL, NULL, NULL);
 double ti = 0.0e0;
 long size = ASet_LocalSize(aset);
 long totalsize = ASet_TotalSize(aset);
 for (long i=0;i<size;i++)
 {
  ti += 0.5*M*(pow(vel[3*i],2.0)+pow(vel[3*i+1],2.0)+pow(vel[3*i+2],2.0));
 }
 ti *= KIN2EV;
 ti /= ((3.0/2.0)*KBOLTZMANN*totalsize);
 ASet_ReturnValue(aset, RET_DOUBLE, &ti, 1);
}
\end{lstlisting}
\end{minipage}}
\caption{The same example of figure~(\ref{fig_code_slot}) but in C.}
\label{}
\end{figure}

To more details about ``slots'' en \textbf{lpmd} 0.7, refeer to the document
``Scheme-Design of LPMD 0.7''.

\section{Using \texttt{Allocate} and \texttt{Deallocate}}

PlugCode incorporate the \texttrademark{Allocate} and \texttt{Deallocate}
functions as replaces of \texttt{malloc}, \texttt{realloc}, and \texttt{free}
used to request and free dynamic memory (as the equivalent \texttt{new} and
\texttt{delete} in C++.


The syntaxis of Allocate are :

\vspace{15pt}
\framebox{
\emph{T}\texttt{ * Allocate (}\emph{T}, \texttt{long} \emph{N}\texttt{)}} \\
\vspace{15pt}

\noindent
the instruction make a memory request for an array of size \emph{N} of
elements of type \emph{T}, and return the pointer to the begining of that
memory. It is similar to the \texttt{malloc} call, in traditional C, however
\texttt{Allocate} keep a track of the requested memory, avoiding memory leaks
produced by the code. And :

\vspace{15pt}
\framebox{
\emph{T}\texttt{ * Allocate (}\emph{T}, \texttt{long} \emph{M}, \emph{T}\texttt{ * }\emph{pointer}\texttt{)}} \\
\vspace{15pt}

\noindent
modify the size of the memory block pointed to \emph{pointer}, in order to
acommodate the size to \emph{M} components of the type \emph{T}. This work in a
similar way than \texttt{realloc} in traditional C, but avoiding memory leaks.

The syntaxis of Deallocate is :

\vspace{15pt}
\framebox{
\texttt{void Deallocate (}\emph{T}\texttt{ * }\emph{pointer}\texttt{)}} \\
\vspace{15pt}

\noindent
this syntaxis free the memory block pointing to the \emph{pointer}
address. It is similar to traditional C, however this keep a tracking bout
the memory block (if it was free previously), avoinding segmentation fauls
by the use of \texttt{free} twice.

\section{The \texttt{Vector} data type}

The \verb'Vector' data type is an aggregation of three real values
(\verb'double' type) sequential in memory. To different uses, is equivalent to
declare as \verb'typedef double Vector[3]'.

\section{The \texttt{VectorLoop} loop}

The \verb'VectorLoop' directive is used to create computation on different
components of a vector in a implicit way. As an example if $a$ and $b$ are
Vector type variables and we want to assign to $b$ the value of $a\times f$
with $f$ a scalar value, then :

\lstset{language=C}
\begin{lstlisting}
double f = 20.0;
VectorLoop { b = a*f; }
\end{lstlisting}

\noindent
that is equivalent in traditional C to :

\lstset{language=C}
\begin{lstlisting}
double f = 20.0;
for (int q=0;q<3;++q) { b[q] = a[q]*f; }
\end{lstlisting}

It is possible use a combination of \verb'Vector' with \verb'VectorLoop' to set
the vector componentes, as in this example :

\lstset{language=C}
\begin{lstlisting}
Vector v;
VectorLoop { v = Vector(1.0, 2.0, 3.0); }
\end{lstlisting}

\noindent
That is equivalent to (in traditional C) :

\lstset{language=C}
\begin{lstlisting}
Vector v;
v[0] = 1.0;
v[1] = 2.0;
v[2] = 3.0;
\end{lstlisting}

\section{Using \texttt{@define}}

The \verb'@define' instruction, can be used to define constant values or
small functions, doing the job of the \verb'#define' macros of standard C.
The next example show the use of \verb'@define' for both cases.

\lstset{language=C}
\begin{lstlisting}
@define KBOLTZMANN = 8.617E-05
@define Suma(x, y) = (x+y)

@slot Test(reader)
{
 double x = Suma(5.0, 3.0*KBOLTZMANN);
}
\end{lstlisting}

\section{Using \texttt{@global}}

The \verb'@global' instruction is used to generate a global variable that is
common to all slots in the code.

\lstset{language=C}
\begin{lstlisting}
@global *ptr(real)

@slot Test1(modifier)
{
 ptr = Allocate(real,size*3);
}

@slot Test2(modifier)
{
 for (int i=0 ; i<size*3 ; ++i)
 {
  ptr[i] = 0.0;
 }
} 
\end{lstlisting}


\section{Using \texttt{@macro}}

The \verb'@macro' instruction define a macro, this mean, a block code (with a
particular name) in which no one to many parameters can be expanded when they
are called.The macros can't return results as functions, despite of course it
is always possbiel return values using specific output parameters. The next
example use a macro to make equal to vectors $v1$ and $v2$.

\lstset{language=C}
\begin{lstlisting}
@macro Igualar(v1, v2)
{
 VectorLoop { v1 = v2; }
}

@slot Prueba2(reader)
{
 Vector a, b;
 Igualar (b, Vector(1,2,3));
 Igualar (a, b);
}
\end{lstlisting}

\section{Simulating value capture by using \texttt{@define} and \texttt{@macro}}

It is possible emulate \emph{clousures} using \texttt{@define} and
\texttt{@macro}. Because both can be expanded as text, it is possible reference
those as ``hanging'' variables (\emph{unbounded}) that are not defined until
they are called. This is a typical examplo of clousure in in which
\verb'@define' a function that add $a=10$ to the argument.

\lstset{language=C}
\begin{lstlisting}
@define SumaDiez(x) = (x+a)
int a = 10;
int y = SumaDiez(5);
assert (y == 15);
a = 20;
y = SumaDiez(5);
assert (y == 25);
\end{lstlisting}

Please note that \verb'SumaDiez' is not a genuine clousure, because: first, it
is not an object but a expanded macro in compiling time (as a template in C++);
second, the $a$ value is not captured inside the function, but take an actual
$a$ value of the previous scope. By changing the value of $a$ to 20,
\verb'SumaDiez(x)' change the result.


\section{References of PlugCode}

\subsection{Data types}

Native (C heritage):

\begin{itemize}
\item \verb'int'
\item \verb'long'
\item \verb'float'
\item \verb'double'
\item \verb'bool'
\item \verb'char'
\end{itemize}

Emulated:

\begin{itemize}
\item \verb'Vector'
\item \verb'Matrix'
\item \verb'AtomPair'
\item \verb'NeighborList'
\item \verb'AtomSelection'
\end{itemize}

``Builders''\footnotemark ~ of emulated types.

\begin{itemize}
\item \verb'Vector Vector(double x, double y, double z)'
\item \verb'Matrix Matrix(int columns, int rows)'
\item \verb'NeighborList NeighborList(AtomPair * pairs, long n)'
\item \verb'AtomSelection AtomSelection(long * indices, long n)'

\end{itemize}

\footnotetext{Strictly speaking C doesn't have a builder concept, \'this are
more like ``factory functions''.}

\subsection{Memory control functions}

\begin{itemize}
\item \verb'T * Allocate(T, long N)'
\item \verb'T * Allocate(T, long N, T * p)'
\item \verb'void Deallocate(T * p)'
\end{itemize}

\subsection{Functions that involve the use of \texttt{Vector}}

\begin{itemize}
\item \verb'const char * VectorFormat(const char * format, Vector v)'
\item \verb'double Module(Vector v)'
\item \verb'double SquareModule(Vector v)'
\item \verb'double Dot(Vector v)'
\item \verb'Vector Cross(Vector a, Vector b)'
\item \verb'Vector Unitary(Vector v)'
\end{itemize}

\subsection{Functions of random number generation}

\begin{itemize}
\item \verb'double Random()'
\item \verb'Vector RandomVector()'
\end{itemize}

\subsection{Functions related to a simulation cell}

\begin{itemize}
\item \verb'double Distance(Vector dr, Vector pi, Vector pj)'
\item \verb'void Fractional(Vector cart, Vector frac)'
\item \verb'void Cartesian(Vector frac, Vector cart)'
\item \verb'void CenterOfMass(Vector cm)'
\item \verb'int InsideNode(Vector v)'
\item \verb'void AddAtom(const char * names, ...)'
\item \verb'void * GetArray(const char * name)'
\item \verb'void GetArrays(const char * names, ...)'
\item \verb'void SetArray(const char * name, long n, void * p)'
\item \verb'void SetArrays(const char * names, long n, ...)'
\item \verb'void * GetTotalArray(const char * name, long a, long b, int sorted)'
\item \verb'void GetTotalArrays(const char * names, long a, long b, int sorted, ...)'
\item \verb'int HasTag(const Tag tag, int key)'
\item \verb'void SetTag(Tag * tag, int key)'
\item \verb'void UnsetTag(Tag * tag, int key)'
\end{itemize}

\subsection{Debug and unit test}

\begin{itemize}
\item \verb'void LogMessage(const char * format, ...)'
\end{itemize}

\subsection{Special variables availables in a ``slot''}

\begin{description}
\item[long size]{: Local size of atoms.}
\item[long totalsize]{: Total size of \'atomos.}
\item[long extsize]{: Extended size of atoms.}
\item[int masternode]{: 1 for masternode 0 otherwise.}
\item[NeighborList neighborlist]{: Precalculated neighbor list.}
\end{description}

\end{document}

